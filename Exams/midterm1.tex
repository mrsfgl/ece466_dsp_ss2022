\documentclass{article}
 
\usepackage[margin=1in]{geometry}
\usepackage{amsmath,amsthm,amssymb}
\usepackage{booktabs}
\usepackage{enumitem,epsfig}
\usepackage{graphicx}
\usepackage{rotating}
\usepackage{multirow}
\usepackage{float}

\def\x{\mathbf{x}}
\def\y{\mathbf{y}}
\def\v{\mathbf{v}}
\def\a{\mathbf{a}}
\def\b{\mathbf{b}}
\def\ttheta{(1-\theta)}
\def\A{\mathbf{A}}
\def\F{\mathbf{F}}
\def\I{\mathbf{I}}
\def\R{\mathbb{R}}
\def\C{\mathcal{C}}
\def\L{\mathcal{L}}

 
\begin{document}
 
\title{ECE 466 Midterm 1}
\author{Name: \\PID: }
 
\maketitle
 
\begin{itemize}
    \item Don't forget to write your name.
    \item Open textbook.
    \item Read carefully and write legibly. For the problems with partial credit, show your work.
    \item For those of you who are remotely solving the exam: 
    \begin{itemize}
        \item You can solve your exam in A-4 sheets or on your tablet.
        \item You need to send a scanned pdf or image until 11:45 AM, Tuesday 22nd, to sofuoglu@msu.edu. Otherwise, your exam will not be accepted.
        \item Make sure your answers are legible from pdf or scanned image. 
    \end{itemize}
\end{itemize}

\begin{enumerate}
%%%%%%%%%%%%%%%%%%%%%%%%%%%%%%%%%%%%%%%%%%%%%%%%%%%%%%%%%%%%%%%%%%%%%%%%%%%%%% 1
    \item  No partial points for the following.
    \begin{enumerate}
        \item {[15 Points]} Check if the following systems fits the classifications on the columns.
         
        
        \begin{tabular}{l|c|c|c|c|c}
            System Equation & Linear & Time Invariant & Static & Causal & Stable\\
            \hline
            $y[n] = x[-n]$ &&&&& \\
            \hline
            $y[n] = 2n^2x[n]+nx[n+1]$ &&&&&\\
            \hline
            $y[n] = cos(2\pi x[n])$ &&&&& \\
        \end{tabular}

        \vspace{1in}

        \item {[5 Points]} The sequence $x[n] = \cos\left(\frac{\pi}{2}n\right)$ was obtained by sampling an analog signal $x(t) = \cos{(\Omega t)}$ at a sampling rate of $F_s=100$ Hz. What are two possible values of $\Omega$? 
        \vspace{1in}
        \item {[5 Points]} What is the ideal sampling frequency of $x(t)=u(t)$? 
        \vspace{1in}
        \item {[5 Points]} The causal sequence $x[n] = \{\underline{3}, 1\}$ is input to a system with impulse response $h[n]$, producing the zero-state response $y[n]=\{\underline{6}, -1, 2, 1\}$. Determine $h[n]$.
        \vspace{2in}
        \item The impulse response of a DT (Discrete Time)-LTI system is given by $h[n] = A (0.7)^nu[n]$. Suppose $x[n] = B \cos(0.2\pi n)u[n]$ is input to the system. Which of the following could be the output signal $y[n] = h[n]\ast x[n]$?
        \begin{enumerate}
            \item $K_1(0.7)^n \cos(0.2\pi n+\theta)u[n]$.
            \item $K_1(0.14)^nu[n] + K_1\cos(0.14 \pi n\theta)u[n]$.
            \item $K_1(0.7)^nu[n]+ K_2\cos(0.2\pi n +\theta)u[n]$.
            \item $K_1(0.7)^nu[-n]+ K_2\cos(0.2\pi n +\theta)u[n]$.
        \end{enumerate}
    \end{enumerate}

    \vspace{1in}
    \item {[30 Points]} Consider a causal LTI system described by the difference equation $y[n] = \frac{2}{15}y[n-1]+\frac{1}{15}y[n-2]+x[n]$ with $y[-1] = 1$, $y[-2]=-1$.
    \begin{enumerate}
        \item {[6]} Find the impulse response $h[n]$.
        \item {[4]} Determine if the system is (1) FIR or IIR, and (2) stable.
        \item {[8]} Find the zero state response for $x[n] = u[n]$. ({\it Decide on particular response's $K$ first.})
        \item {[8]} Find the zero input response.
        \item {[4]} Find the total response for $x[n]=u[n]$. Identify the steady state and transient responses.
    \end{enumerate}
    \newpage
    \textit{Extra page for Question 2}
    \newpage
    \item {[30 points]} A causal LTI system has a system function $H(z) = \frac{1+z^{-1}}{1-\frac{3}{5}z^{-1}+\frac{2}{25}z^{-2}}.$
    \begin{enumerate}
        \item {[5]} Determine the difference equation that this system function describes.
        \item {[2]} What is the gain of the system?
        \item {[5]} Plot the pole-zero map.
        \item {[5]} Determine the region of convergence (ROC).
        \item {[5]} Is the system stable? Why?
        \item {[8]} Find the input signal $x[n]$ that will produce the output $y[n] = 2\left(\frac{2}{5}\right)^nu[n]-\left(\frac{1}{5}\right)^nu[n]$.
    \end{enumerate}
    
    \newpage
    \textit{Extra page for Question 3}
\end{enumerate}

\end{document}